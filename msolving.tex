%------------------------------------------------------------
\lecture{Multi-shot ASP Solving}{msolving}
%------------------------------------------------------------
\part{Multi-shot ASP Solving}
%------------------------------------------------------------
%\newcommand{\PRG}{\ensuremath{P}}
\newcommand{\QRG}{\ensuremath{Q}}
\newcommand{\RRG}{\ensuremath{R}}
%------------------------------------------------------------
\section{Motivation}
%------------------------------------------------------------
\begin{frame}{Motivation}
  \begin{itemize}
  \item<1-> \structure{Multi-shot solving} allows for solving continuously changing logic programs in an operative way
    \medskip
    \begin{itemize}\normalsize
    \item<2-> \alert<2-3>{Single-shot solving}\pause[3]: \
      \textit{ground}\ $|$ \textit{solve}\
      \medskip
    \item<4-> \alert<4->{Multi-shot solving}: \
      \only<6->{$($\,}\only<7->{\textit{input} $|$ }\textit{ground}\only<5->{$^*$}\;$|$\;\textit{solve}\only<5->{$^*$}\only<8->{$|$ \textit{theory} }\only<9->{$|$ \dots}\only<6->{$)^*$}
    \end{itemize}
    \medskip
  \item<10-> \structure{Application areas}
  \item<10-> [] Agents, Assisted Living, Robotics, Planning, Query-answering, etc
    \bigskip
  \item <11-> \structure{Idea} \ \clingo\ \ = \ ASP + Control
    \smallskip
    \begin{itemize}\normalsize
    \item Extend ASP with dedicated directives
      \smallskip
    \item Provide powerful API (here: \python)
    \end{itemize}
  \end{itemize}
\end{frame}
%
%%% Local Variables:
%%% mode: latex
%%% TeX-master: "../asp"
%%% End:

% ----------------------------------------------------------------------
\begin{frame}[c]{Clingo \ = \ ASP + Control}
  \begin{itemize}
  \item<2-> \structure{ASP}
    \begin{itemize}
    \item<2-> \texttt{\#program <name> [ (<parameters>) ]}
      \begin{itemize}
      \item<3-> \structure{Example}   \texttt{\#program play(t).}
%      \item \structure{Default}\, \ \texttt{\#program base.}
      \end{itemize}
    \item <2->\texttt{\#external <atom> [ : <body> ]}
      \begin{itemize}
      \item<3-> \structure{Example} \texttt{\#external mark(X,Y,P,t)\;:\;field(X,Y),\;player(P).}
      \end{itemize}
    \end{itemize}
    \medskip
  \item<4-> \structure{Control}
    \begin{itemize}
    % \item<4-> Lua \ (\texttt{www.lua.org})
    %   \begin{itemize}
    %   \item<5-> \structure{Example} \texttt{prg:solve(), prg:ground(parts), \dots}
    %   \end{itemize}
    \item<4-> Python \ (\texttt{www.python.org})
      \begin{itemize}
      \item<5-> \structure{Example} \texttt{prg.solve(), prg.ground(parts), \dots}
      \end{itemize}
    \item <6-> C, Lua, and Prolog embeddings are available too
    \end{itemize}
    \medskip
    \item<7-> \structure{Integration}
      \begin{itemize}
      \item <7-> in ASP: embedded scripting language \ (\texttt{\#script})
      \item <7-> in Python: library import \ (\texttt{import clingo})
      \end{itemize}
  \end{itemize}
\end{frame}
% ----------------------------------------------------------------------
\begin{frame}[fragile,c]{Vanilla \textit{clingo}}
  \pause
  \begin{center}
    \begin{minipage}{0.5\linewidth}
\begin{semiverbatim}
\alert<2>{#script} (python)
def \alert<3>{main}(prg):
    parts = []
    parts.append(("base", []))
    prg.\alert<3>{ground}(parts)
    prg.\alert<3>{solve}()
\alert<2>{#end}.
\end{semiverbatim}
    \end{minipage}
  \end{center}
\end{frame}
% ----------------------------------------------------------------------
\begin{frame}[fragile,c]{Hello world!}

\begin{center}
\begin{minipage}{0.5\linewidth}
\begin{semiverbatim}
{\bf{}#script} (python)
def \alert<3>{main}(prg):
    print("Hello world!")
{\bf{}#end}.
\end{semiverbatim}
\end{minipage}

\bigskip\pause\bigskip

\begin{minipage}{0.95\linewidth}\footnotesize
\begin{semiverbatim}
$ clingo hello.lp\pause
clingo version 4.5.0
Reading from hello.lp
\alert<4>{Hello world!}
UNKNOWN

Models       : 0+
Calls        : 1
Time         : 0.009s (Solving: 0.00s 1st Model: 0.00s Unsat: 0.00s)
CPU Time     : 0.000s
\end{semiverbatim}
\end{minipage}
\end{center}
\end{frame}
% ----------------------------------------------------------------------
\begin{frame}[fragile,c]{Preview on incremental solving}
\begin{lstlisting}
#(\pause#)#program base.

p(0).#(\pause#)

#program step (t).

p(t) :- p(t-1).#(\pause#)

#program check (t).
#external plug(t).

:- not p(42), plug(t).
\end{lstlisting}
\end{frame}
% ------------------------------------------------------------
\section{\texttt{\#program} and \texttt{\#external} declaration}
%------------------------------------------------------------
\begin{frame}[fragile]{\texttt{\#program} declaration}
  \begin{itemize}
  \item A \alert{program declaration} is of form
    \[
    \texttt{\#program}~n\,(p_1,\dots,p_k)
    \]
    where $n,p_1,\dots,p_k$ are non-integer constants
  \item <2-> We call $n$ the \alert<2>{name} of the declaration and $p_1,\dots,p_k$ its \alert<2>{parameters}
    \smallskip
  \item <3-> \structure{Convention}
    Different occurrences of program declarations with the same name share the same parameters
    % \only<4>{(yet \clingo~4 permits others)}
    \smallskip
  \item <5-> \structure{Example} \
    \begin{minipage}[t]{0.5\linewidth}
      \begin{semiverbatim}
        #program \alert<5>{acid(k)}.
        b(k).
        c(X,k) :- a(X).
        #program \alert<5>{base}.
        a(2).
      \end{semiverbatim}
    \end{minipage}
  \end{itemize}
\end{frame}
%------------------------------------------------------------
\begin{frame}[fragile]{Scope of \texttt{\#program} declarations}
  \begin{itemize}
  \item <1->
    The \alert{scope} of an occurrence of a program declaration in a list of rules and declarations
    consists of the set of all rules and non-program declarations appearing between
    the occurrence and the next occurrence of a program declaration or the end of the list
  \item <2->
    Rules and non-program declarations outside the scope of any program declaration
    are implicitly preceded by a \alert<2>{\lstinline{base}} program declaration
    \medskip
  \item <3-> \structure{Example} \
    \begin{minipage}[t]{0.5\linewidth}
      \begin{semiverbatim}
        \alert<5>{a(1).}
        #program \alert<4>{acid(k)}.
        \alert<4>{b(k).}
        \alert<4>{c(X,k) :- a(X).}
        #program \alert<5>{base}.
        \alert<5>{a(2).}
      \end{semiverbatim}
    \end{minipage}
  \end{itemize}
\end{frame}
%------------------------------------------------------------
\begin{frame}{Scope of \texttt{\#program} declarations}
  \begin{itemize}
  \item <1->
    Given a list $\RRG$ of (non-ground) rules and declarations and a name $n$,
    we define $\RRG(n)$ as the set of all rules and non-program declarations
    in the scope of all occurrences of program declarations with name $n$
  \item <2->
    We often refer to $\RRG(n)$ as a subprogram of $\RRG$
    \medskip
  \item <3-> \structure{Example} \
    \begin{itemize}\normalsize
    \item \(
      \RRG(\texttt{base})=\{a(1),a(2)\}
      \)
      \smallskip
    \item
      \(
      \RRG(\texttt{acid})\only<5->{[k/42]}=\{b(\only<-5>{k}\only<6>{42}), c(X,\only<-5>{k}\only<6>{42})\leftarrow a(X)\}\only<5>{[k/42]}
      \)
    \end{itemize}
      \smallskip
  \item <4->
    Given a name $n$ with associated parameters $(p_1,\dots,p_k)$,
    the instantiation of $\RRG(n)$ with a term tuple $(t_1,\dots,t_k)$
    results in the set
    \[
    \RRG(n)[p_1/t_1,\dots,p_k/t_k]
    \]
    obtained by replacing in $\RRG(n)$ each occurrence of $p_i$ by $t_i$ % for $1\leq i\leq k$
  \end{itemize}
\end{frame}
% ------------------------------------------------------------
\begin{frame}{Contextual grounding}
  \begin{itemize}
  \item <1-> Rules are grounded relative to a set of atoms, called \alert{atom base}
  \item <2->
    Given a set \RRG\ of (non-ground) rules and two sets $C,D$ of ground atoms,
    we define an instantiation of \RRG\ relative to $C$ as a ground program \cground{\RRG}{C} over $D$
    subject to the following conditions:
    \begin{align*}
      C& {}\subseteq D\subseteq C \cup\head{\cground{\RRG}{C}}
      \\
      \cground{\RRG}{C}& {} \subseteq
                         \{
                         \head{r}\leftarrow\pbody{r}\cup \{\naf{a} \mid a\in\nbody{r}\cap D\}
      \\&\qquad\qquad\qquad\qquad
          \mid
          r\in \ground{\RRG},
          \pbody{r}\subseteq D
          \}
    \end{align*}
  \item <3-> \structure{Example} Given
    \(
    R= \{\; a(X) \leftarrow f(X), e(X);\ b(X) \leftarrow f(X), \naf{e(X)} \;\}
    \)

    and
    \(
    C=\{f(1),f(2),e(1)\}
    \),
    we obtain
    \[
    \cground{R}{C}
    =
    \left\{
      \begin{array}{ll}
        a(1) \leftarrow f(1), e(1); & b(1) \leftarrow f(1), \naf{e(1)}\\
                                    & b(2) \leftarrow f(2)
      \end{array}
    \right\}
    \]
  \end{itemize}
\end{frame}
% ------------------------------------------------------------
\begin{frame}[fragile]{\texttt{\#external} declaration}
  \begin{itemize}
  \item<1-> An \alert{external declaration} is of form
    \[ % begin{equation}\label{eq:external:declaration}
      \texttt{\#external}~a : B
    \] % end{equation}
    where $a$ is an atom and $B$ a rule body
  \item <2->
    A logic program with external declarations is said to be \alert<2>{extensible}
    \medskip
  \item <3-> \structure{Example} \
    \begin{minipage}[t]{0.5\linewidth}
      \begin{semiverbatim}
        #external e(X) : f(X), X < 2.
        f(1..2).
        a(X) :- f(X), e(X).
        b(X) :- f(X), not e(X).
      \end{semiverbatim}
    \end{minipage}
  \end{itemize}
\end{frame}
% ------------------------------------------------------------
\begin{frame}[fragile]{Grounding extensible logic programs}
  \begin{itemize}
  \item <1->
    Given an extensible program $\RRG$,
    we define
    \begin{align*}
      \QRG & {} = \{a\leftarrow B,\varepsilon\mid({\small\textbf{\texttt{\#external}}}~a : B)\in \RRG\}\\
      \RRG'& {} = \{a\leftarrow B\in\RRG\}
    \end{align*}
  \item <2-> \structure{Note} \
    An external declaration is treated as a rule \
    \(
    a \leftarrow B,\varepsilon
    \)\\
    where $\varepsilon$ is a ground marking atom
  \item <3->
    Given an atom base $C$,
    the ground instantiation of an extensible logic program $\RRG$ is defined as
    a (ground) \alert{logic program}~$P$ \alert{with externals}~$E$
    where
    \begin{align*}
      P & {} = \{r\in\cground{\RRG'\cup \QRG}{C\cup\{\varepsilon\}} \mid \varepsilon\notin\body{r}\}\\
      E & {} = \{\head{r}\mid r\in\cground{\RRG'\cup \QRG}{C\cup\{\varepsilon\}}, \varepsilon\in\body{r}\}
    \end{align*}
  \item <4-> \structure{Note} The marking atom $\varepsilon$ appears neither in $P$ nor $E$, respectively,
    \phantom{Note} and $P$ is a logic program over $C\cup E\cup\head{P}$
  \end{itemize}
\end{frame}
% ------------------------------------------------------------
\begin{frame}[fragile]{Example}
  \begin{itemize}
  \item Extensible program
    \begin{semiverbatim}
  \only<1-2>{\only<1>{#external }e(X) :\only<2->{-} f(X), g(X)\only<2->{, \(\varepsilon\)}.}\only<3->{e(1) :- f(1), g(1), \(\varepsilon\).}\only<3-7>{   \textcolor<7>{red}{e(2)} :- f(2), \textcolor<6->{red}{g(2)}, \(\varepsilon\).}
  f(1). f(2).
  \only<1-3>{a(X) :- f(X), e(X).}\only<4->{a(1) :- f(1), e(1).}\only<4-7>{      a(2) :- f(2), \textcolor<7>{red}{e(2)}.}
  \only<1-3>{b(X) :- f(X), not e(X).}\only<4->{b(1) :- f(1), not e(1).}\only<4->{  b(2) :- f(2)\only<-7>{, not \textcolor<7>{red}{e(2)}}.}
    \end{semiverbatim}
    \vspace{-3ex}
  \item[]<5-> Atom base $\{\text{\lstinline{g(1)}}\}\cup\{\varepsilon\}$
    \medskip
  \item<9-> Ground program
    \begin{semiverbatim}
  f(1). f(2).
  a(1) :- \only<-9>{f(1), }e(1).
  b(1) :- \only<-9>{f(1), }not e(1).\only<10>{      }   b(2)\only<-9>{ :- f(2)}.
    \end{semiverbatim}
    \vspace{-3ex}
  \item[]<9-> with externals $\{\text{\lstinline{e(1)}}\}$
  \end{itemize}
\end{frame}
% ------------------------------------------------------------
% \section{Interaction}
%%% Local Variables:
%%% mode: latex
%%% TeX-master: "asp"
%%% End:
