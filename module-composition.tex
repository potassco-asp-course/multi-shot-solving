\newcommand{\QRG}{\ensuremath{Q}}
\newcommand{\RRG}{\ensuremath{R}}
% ------------------------------------------------------------
\begin{frame}[c]{Module}
  \begin{itemize}
  \item<1-> The assembly of subprograms can be characterized\\ by means of modules:
    \bigskip
  \item<2-> A \alert{module} \module{P} is a triple
    \(
    (P,I,O)
    \)
    consisting of
    \begin{itemize}\normalsize
    \item a (ground) program $P$ over $\ground{\mathcal{A}}$
      and
    \item sets $I,O\subseteq\ground{\mathcal{A}}$ such that
      \begin{itemize}\normalsize
      \item $I\cap O=\emptyset$,
      \item $\atom{P}\subseteq I\cup O$, and
      \item $\head{P}\subseteq O$
      \end{itemize}
    \end{itemize}
    \medskip
  \item<3-> The elements of~$I$ and~$O$ are called \alert{input} and \alert{output atoms}
    \begin{itemize}
    \item<4-> denoted by $I(\module{P})$ and $O(\module{P})$
    \end{itemize}
  \item<5-> Similarly, we refer to (ground) \alert{program}~$P$ by $P(\module{P})$
  \end{itemize}
  % We say that $\module{P}$ is \alert{input-free}, if $I(\module{P})=\emptyset$
\end{frame}
% ------------------------------------------------------------
\begin{frame}{Composing modules}
  \begin{itemize}
  \item<2-> Two modules~\module{P} and~\module{\QRG} are
    \alert{compositional}\pause[3], if
    \begin{itemize}
    \item <3->
      $\out{\module{P}}\cap\out{\module{\QRG}}=\emptyset$
      \pause[4]and
      \smallskip
    \item <4->
      $\out{\module{P}}\cap S=\emptyset$ or
      $\out{\module{\QRG}}\cap S=\emptyset$
      \\
      for every strongly connected component~$S$ of
      $\prog{\module{P}}\cup\prog{\module{\QRG}}$
    \end{itemize}
  \item<5-> \structure{Note}
    \begin{itemize}
    \item Recursion between two modules to be joined is disallowed
      \smallskip
    \item Recursion within each module is allowed
    \end{itemize}
    \medskip
  \item<6-> The \alert{join}, $\module{P} \sqcup \module{\QRG}$, of two modules
    $\module{P}$ and $\module{\QRG}$ is defined as the module
    \[
    (\; \alert<2>{P(\module{P}) \cup P(\module{\QRG})}\,,
     \; \alert<3>{(I(\module{P})\setminus O(\module{\QRG}))
                  \cup
                  (I(\module{\QRG})\setminus O(\module{P}))}\,,
     \; \alert<4>{O(\module{P}) \cup O(\module{\QRG})}
     \;)
    \]
    \alert<7>{provided that~\module{P} and~\module{\QRG} are compositional}
  \end{itemize}
\end{frame}
% ------------------------------------------------------------
\begin{frame}{Composing logic programs with externals}
  \begin{itemize}
  \item<1-> \structure{Idea} Each \lstinline{ground} instruction induces a module to be joined\\ with the module representing the current program state
    \medskip
  \item<2->
    Given an atom base $C$,
    a (non-ground) extensible program \RRG\\ induces the module
    \[ % begin{equation}\label{eq:ground:module}
      \inst{\module{\RRG}}{C}
      =
      (P,(C\cup E)\setminus\head{P},\head{P})
    \] % end{equation}
    via the ground program $P$ with externals $E$ obtained from \RRG\ and $C$
    \medskip
  \item<3-> \structure{Note}
    $E\setminus\head{P}$ consists of atoms stemming from non-overwritten\\ external declarations
\end{itemize}
\end{frame}
% ------------------------------------------------------------
\begin{frame}[fragile]{Example}
  \begin{itemize}
  \item<1-> Atom base $C=\{\text{\lstinline{g(1)}}\}$
  \item<only@1,4> Extensible program $\RRG$
    \begin{semiverbatim}
  #external e(X) : f(X), g(X)
  f(1). f(2).
  a(X) :- f(X), e(X).
  b(X) :- f(X), not e(X).
    \end{semiverbatim}
    \vspace{-10pt}
  \item<only@2-3> Ground program $P$
    \begin{semiverbatim}
  f(1). f(2).
  a(1) :- f(1), e(1).
  b(1) :- f(1), not e(1).   b(2) :- f(2).
    \end{semiverbatim}
    \vspace{-13pt}
  \item<only@2-3>[] with externals $E=\{\text{\lstinline{e(1)}}\}$
  \item<3-> Module
    \(
    \inst{\module{\RRG}}{C}=(P,(C\cup E)\setminus\head{P},\head{P})
    \)
    \[
    \hspace{-15pt}
    {} =
    \left(
      \left\{
        \begin{array}{l}
          f(1), \  f(2), \\
          a(1) \leftarrow f(1), e(1),\\
          b(1) \leftarrow f(1), \naf{e(1)},\\
          b(2) \leftarrow f(2)
        \end{array}
      \right\}
      ,
      \left\{
        \begin{array}[l]{l}
          g(1),\\e(1)
        \end{array}
      \right\}
      ,
      \left\{
        \begin{array}[l]{l}
          f(1), \  f(2), \\
          a(1),\\
          b(1), b(2)
        \end{array}
      \right\}
    \right)
    \]
  \end{itemize}
\end{frame}
% ------------------------------------------------------------
\begin{frame}{Capturing program states by modules}
  \begin{itemize}
  \item<1-> Each program state is captured by a module
    \begin{itemize}
    \item<2-> The input and output atoms of each module provide the atom base
    \end{itemize}
    \medskip
  \item<3-> The initial program state is given by the empty module
    \[
    \module{P}_0 = (\emptyset,\emptyset,\emptyset)
    \]
  \item<4->
    The program state succeeding~$\module{P}_i$ is captured by the module
    \[ % begin{equation}\label{eq:module:composition}
      \module{P}_{i+1}
      =
      \module{P}_i
      \sqcup
      \inst{\module{\RRG}_{i+1}}{\inp{\module{P}_i}\cup\out{\module{P}_i}}
    \] % end{equation}
    where
    \(
    \inst{\module{\RRG}_{i+1}}{\inp{\module{P}_i}\cup\out{\module{P}_i}}
    \)
    captures the result of grounding an extensible program \RRG\ relative to atom base $\inp{\module{P}_i}\cup\out{\module{P}_i}$
    \smallskip
  \item<5-> \structure{Note} The join leading to $\module{P}_{i+1}$ can be undefined in case the constituent modules are non-compositional
  \end{itemize}
\end{frame}
% ------------------------------------------------------------
\begin{frame}{Capturing program states by modules}
  \begin{itemize}
  \item<1->
Let $(\RRG_i)_{i>0}$ be a sequence of (non-ground) extensible programs,
and let
$P_{i+1}$ be the ground program with externals $E_{i+1}$ obtained from $\RRG_{i+1}$ and $\inp{\module{P}_i}\cup\out{\module{P}_i}$
\bigskip
\item<2->[]
If
\(
\bigsqcup_{i\geq 0}\module{P}_i
\)
is compositional,
then
\medskip
\begin{enumerate}\normalsize
\item
\(
\prog{\bigsqcup_{i\geq 0}\module{P}_i}
=
\bigcup_{i>0} P_i
\)
\smallskip
\item
\(
\;\inp{\bigsqcup_{i\geq 0}\module{P}_i}
=
\bigcup_{i>0} E_i\setminus\bigcup_{i>0} \head{P_i}
\)
\smallskip
\item
\(
\out{\bigsqcup_{i\geq 0}\module{P}_i}
=
\bigcup_{i>0} \head{P_i}
\)
\end{enumerate}
\end{itemize}
\end{frame}
